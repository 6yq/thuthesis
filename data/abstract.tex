% !TeX root = ../thuthesis-example.tex

% 中英文摘要和关键字

\begin{abstract}
  中微子实验依赖 PMT 捕获物理事例产生的少量光子,来对事例进行位置与时间重建,因此 PMT 的增益刻度对提高探测器的能量分辨率具有重要意义。
  包括 JUNO 在内的部分中微子探测器使用由北方夜视生产的 MCP-PMT 作为主要的光电探测器件。
  该新型号的 MCP-PMT 在使用 ALD 镀膜增大光电子在 MCP 表面的二次倍增系数与 PMT 的收集效率的同时,
  也为单光电子电荷响应带来了与众不同的“长尾”的结构,为该种 PMT 的增益刻度与实际使用带来了困难。

  本研究将于 8 寸 MCP-PMT 提出的 Gamma-Tweedie 混合电荷模型用于 20 寸型号的刻度,验证了其具有相似的性质,
  并对刻度方法加入了光强参数项,提高了对高电荷道址区域的拟合准确度,使得在高光强工作环境下拟合该 MCP-PMT 电荷谱变得可能。
  
  本研究的电荷基于波形分析方法 FSMP 获得,并寻找了由刻度结果从 Gamma-Tweedie 混合电荷模型在极大似然意义下的多高斯混合模型近似,
  使得波形分析的方法迭代变为现实,并也将提高增益刻度的准确度,最终有效提高 JUNO 的能量分辨率。

  % 关键词用“英文逗号”分隔,输出时会自动处理为正确的分隔符
  \thusetup{
    keywords = {光电倍增管, 微通道板,增益刻度, 电荷模型, 高斯混合模型},
  }
\end{abstract}

\begin{abstract*}
    Neutrino experiments rely on PMTs to capture photons generated by physical events, aiming to reconstruct the location and the time. 
    Therefore the gain calibration of PMT is of great significance to improve the energy resolution of the detector.
    Some neutrino detectors, including JUNO, use the MCP-PMT produced by North Night Vision as the main detecting force.
    This brand-new type of MCP-PMT applies ALD coating to increase the multiplication factor of photoelectrons on the MCP surface, as well as the collection efficiency of PMT.
    It also brings a unique `long tail' structure to the single photoelectron charge response spectrum, 
    which makes it difficult to calibrate and use.

    In this study, the Gamma-Tweedie mixture charge model proposed with 8-inch MCP-PMT is applied to 20-inch ones and indicated similar behaviors.
    Furthermore, light intensity is taken into consideration of the calibration method, 
    thus improving the fitting accuracy of the high charge area entries,
    also making possible utilization under intensive light circumstances.

    The charge spectrum of this work is obtained from waveform analysis method FSMP, 
    and the best Gaussian Mixture Model is found as a maximum-likelihood approximation to the Gamma-Tweedie mixture charge model.
    The iterative method of waveform analysis becomes a reality, 
    and the accuracy improvement of gain calibration could be expected, finally the energy resolution of JUNO as well.

  % Use comma as separator when inputting
  \thusetup{
    keywords* = {PMT, MCP, gain calibration, charge model, Gaussian Mxiture Model},
  }
\end{abstract*}
