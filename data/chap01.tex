% !TeX root = ../thuthesis-example.tex

\chapter{引言}

\section{中微子实验}

\section{江门中微子实验与 OSIRIS 探测器}

\section{新型光电倍增管的特性}
光电倍增管是大型中微子实验的核心器件,它主要用来捕获光信号,来对中微子在液体闪烁体或者水中发生相互作用后释放的光子进行探测。
PMT 相较于半导体探测器等其他探测器件,具有灵敏度高的特点:由于产生载流子需要的能量并不高,
且 PMT 的多级放大能够有效倍增载流子的数量,即使光子数量非常少,
也能够有效地产生大量电子,在 PMT 阳极上接收到显著的电压信号。

为了解决依赖国外公司 PMT 进口的高额成本问题,
中国科学院高能物理研究所、北方夜视技术股份有限公司、中国科学院西安光学精密机械研究所等单位合作
设计、研究、改进与生产了新型微通道板型光电倍增管(Microchannel Plate Photomultiplier Tubes,以下简称 MCP-PMT),
在其量子效率(Quantum Efficiency,以下简称 QE)与收集效率(Collection Efficiency,以下简称 CE)等方面均做出了改进并有效降低了成本。

该 MCP-PMT 的突出特点包括:
\begin{enumerate}
    \item 相较于常见的打拿级 PMT,使用具有微小倾角的斜长微通道板(Microchannel Plate,以下简称 MCP)取代了分离式的多级打拿级,
    因此可以发生倍增物理过程的接触点变得连续,PMT 外接高压引线结构得到简化,电子的收集效率也较高;
    \item 相对于其他种类的 MCP-PMT,该 PMT 的特点为在 MCP 表面引入了主要成分为复合 $\text{Al}_2\text{O}_3-\text{MgO}$ 的
    原子沉积涂层(atomic layer deposition,以下简称 ALD),该种 ALD 涂层具有高二次电子倍增系数的特点,
    即电子在该表面具有较高的激发二次电子的概率与二次电子数目的期望;
    \item 单位数目的电子在入射 MCP-PMT 后,能够产生更多的倍增电子并在阳极被收集(比值定义为 CE),
    因此理论意义上具有更优秀的峰谷比与分辨率;
    \item 由于二次倍增电子的存在,收集到大电荷信号的概率较其他 PMT 高,表现在电荷谱具有“长尾”结构。
\end{enumerate}

其中 2. 与 3. 为该 MCP-PMT 的突出优点,并在测试\cite{zhangPerformanceEvaluation8inch2023} 中得到了验证。
同时,相较于传统 PMT 可以使用高斯分布描述的对称型单光电子响应电荷谱概率密度分布,4. 呈现的非对称长尾结构为使用该 PMT 带来了困难:
\begin{itemize}
    \item 分压实验研究\cite{yangMCPPerformanceImprovement2017} 中发现 MCP 对低能电子的增益远小于高能电子,揭示了单光电子电荷谱不能够认为只有单一的增益模式;
    \item 只截取主峰部分使用高斯分布拟合,则不能够充分利用长尾部分的信息,对提高统计量与能量分辨率没有帮助,
    且在没有物理机理认识的前提下直接使用主峰峰值定义增益不具有充分的说服力;
    \item 在光强较强时,电荷谱将有多个峰结构,较大电荷由长尾与整数倍主峰共同贡献,使得能量估计有偏。
\end{itemize}

在 JUNO 中央探测器上约有 75\% 的大 PMT 使用了该新型款式,
在先行探测器 OSIRIS 上安装的 76 个大 PMT 也均为该型号。
相较于其他应用传统 PMT 的中微子探测器,JUNO 与 等探测器需要对新 PMT 展开仔细地刻度等研究,
增进对该类型 PMT 的理解与经验,尤其需要利用适合的电荷模型完成刻度,为 PMT 的信号读出提供坚实的物理基础,
进而达到提升能量分辨率的目标。

\section{论文工作总结}
