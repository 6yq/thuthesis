% !TeX root = ../thuthesis-example.tex

\chapter{结合波形分析的联合刻度方法}

\section{分区间数据的拟合方法}
由于一次激光刻度取数中,电荷计数能够达到数万甚至数十万,
直接对每个计数进行拟合的内存消耗与效率极不理想。
如果能够以尽可能接近电荷模型原始概率分布的分区间方式,
将需要计算的计算点减少至数十至数百个,
将有约三个数量级的效率提升。

\subsection{分区间方法}
常用的分区间包括平方根方法、Sturges 公式、Rice 准则、Scott 准则、Doane 公式、Freedman-Diaconis 准则等。

\begin{table}
    \centering
    \caption{多种分区间方法的比较}
    \begin{tabular}{llll}
    \toprule
    方法                   & 区间数                                                                                                                                                                                                                            & 区间宽                                                                                                                                                    & 优势与适用范围                     \\
    \midrule
    平方根方法                & $k=\lceil\sqrt{n}\rceil$                                                                                                                                                                                                       & -                                                                                                                                                      & 简单                          \\
    Sturges 公式           & $k=\lceil\log_2n\rceil+1$                                                                                                                                                                                                      & -                                                                                                                                                      & 正态分布;样本数量适中                 \\
    Rice 准则              & $k=\lceil2\sqrt[3]{n}\rceil$                                                                                                                                                                                                   & -                                                                                                                                                      & 简单                          \\
    Scott 准则             & -                                                                                                                                                                                                                              & $h=\frac{3.49\hat{\sigma}}{\sqrt[3]{n}}$                                                                                                               & 简单;正态分布                     \\
    Terrell-Scott 准则     & $k=\sqrt[3]{2n}$                                                                                                                                                                                                               & -                                                                                                                                                      & 渐进极小均方误差的积分;适用于非正态分布        \\
    Doane 公式             & $k=1+\log_2(n)+\log_2\left(1+\frac{|g_1|}{\sigma_{g_1}}\right)$,$g_1$为偏度,$\sigma_{g_1}=\sqrt{\frac{6(n-2)}{(n+1)(n+3)}}$ & -                                                                                                                                                      & 基于 Sturges 公式;使用偏度,适用于非正态分布 \\
    Freedman-Diaconis 准则 & -                                                                                                                                                                                                                              & $h=2\frac{\mathrm{IQR}(x)}{\sqrt[3]{n}}$,IQR 为四分位距 & 对离群值不敏感;考虑偏度,适用于非正态分布       \\
    \bottomrule
    \end{tabular}\label{tab:histogram}
\end{table}

在表~\ref{tab:histogram} 中,Freedman-Diaconis 准则既在系数部分使用了四分位距 IQR,
考虑了随机变量分布的偏度而避免像 Doane 公式一样直接计算偏度与其标准差,降低了计算难度,
又在幂次上同样渐进极小均方误差的积分,
即 $k\propto n^{1/3}\ \simeq\ h\propto n^{-1/3}$\cite{freedmanHistogramDensityEstimator1981},
确保在样本统计量大的情况下能够以最优的幂次还原原始概率分布。
综上所述,本研究中使用 Freedman-Diaconis 准则来对电荷计数进行分区间,再进行拟合。

\subsection{最小二乘法}
不妨约定 $q$ 为自变量,$q_i$ 表示第 $i$ 个直方图区间的自变量闭区间,
$n_i$ 表示第 $i$ 个直方图区间的实际计数,或称为频数,总事例数 $N=\sum_{i}n_i$;
$e_i$ 表示第 $i$ 个直方图区间的预期计数,或称为预期频数,可由下式给出:
\begin{equation}
    e_i = N\int_{\inf q_i}^{\sup q_i}\mathrm{d}q\cdot f(q;\boldsymbol{\theta})
\end{equation}

其中 $f(q;\boldsymbol{\theta})$ 代表在给定参数集 $\theta$ 时电荷谱的条件概率密度函数。

将直方图的每一个区间都视为一次测量,测量的结果为频数,
因此可以定义 $\chi^2(\boldsymbol{\theta})$:
\begin{equation}
    \chi^2(\boldsymbol{\theta})\equiv\sum_{i}\frac{\left[n_i-e_i(\boldsymbol{\theta})\right]^2}{\sigma_i^2}
\end{equation}

其中 $\sigma_i^2$ 代表第 $i$ 个直方图区间的频数的方差。
为了计算方差,使用预期频数 $e_i$ 替代方差 $\sigma_i^2$ 或使用频数 $n_i$ 替代方差 $\sigma_i^2$ 是两种可行的做法。

如果每个区间的计数 $n_i$ 相较总事例数 $N$ 占比小,可以认为渐进总事例数 $N$ 无穷、发生概率 $p$ 无穷小而 $Np$ 有限的泊松分布 $\pi(n_i;Np)$,
因此有近似 $\sigma^2_i=n_i$:
\begin{equation}
    \chi^2(\boldsymbol{\theta})\equiv\sum_{i}\frac{\left[n_i-e_i(\boldsymbol{\theta})\right]^2}{n_i}
    \label{eq:pois-ls}
\end{equation}

如果将每个区间视为各自独立且具有较大数学期望的计数结果,则可近似认为服从高斯分布,
因其误差可以认为仅仅由统计涨落贡献,故具有方差与数学期望相等的特征,
即 $\sigma^2_i=e_i$:
\begin{equation}
    \chi^2(\boldsymbol{\theta})\equiv\sum_{i}\frac{\left[n_i-e_i(\boldsymbol{\theta})\right]^2}{e_i}
    \label{eq:gauss-ls}
\end{equation}

根据大数定律,这两种近似在样本量趋近无穷时,具有相同的极限,但它们在有限样本量下的取舍则有所不同:
\begin{enumerate}
    \item~\eqref{eq:pois-ls} 需要每个区间事例数占比都较小;
    \item~\eqref{eq:gauss-ls} 需要每个区间事例数足够多,尤其不能为空,否则不能良定义。
\end{enumerate}

\subsection{极大似然法}


\section{电荷谱的获取}

\subsection{波形分析}


\subsection{时间窗筛选}


\section{傅里叶变换光强刻度}

处于三个原因,考虑光强的刻度方法有被引入的必要性:
\begin{enumerate}
    \item 在 PMT 的刻度与实际运行中,很可能遇到光强较强以至于多个光电子情形不可忽略甚至占主要成分的情况;
    \item 由于本研究所考虑的 MCP-PMT 相较于传统打拿级 PMT,具有电荷长尾的效应,
    实际电荷谱上较主峰能道高的区域由单光电子模型中 Tweedie 成分与多光电子的主峰共同贡献;
    \item 对于低光强的刻度数据,大统计量的 0 电荷计数被排除在单光电子谱刻度外,没有得到充分利用。
\end{enumerate}

拟合引入 0 电荷计数,能够极大提高统计量,得到相对误差较小的光强,
从而在电荷谱的能道较高范围内尽可能消除多光电子的贡献对长尾参数的影响,
提高目标 MCP-PMT 的增益刻度分辨率。

\subsection{任意光电子电荷谱}\label{sec:dft}

光电子个数服从期望为 $\mu$ 的泊松分布,即:
\begin{equation}
    P(\text{PE}=n) = \pi(n;\mu)=\sum_{i=0}^{+\infty}\frac{\mu^{n}}{n!}\cdot e^{-\mu}
\end{equation}

当光电子个数为 0 时,考虑到没有光电子时的电荷应该严格为 0,电荷谱密度可以认为是:
\begin{equation}
    S_{0\text{PE}}(q)=N_{0\text{PE}}\cdot\delta(q)=\pi(0;\mu)\cdot\delta(q)
\end{equation}

当光电子个数为正整数 $k\in\mathbb{N}^{+}$ 时,电荷谱密度等效于 $k$ 个单光电子谱卷积:
\begin{align}
    S_{k}(Q = \sum_{i = 1}^{k} q_i ) 
    & = \pi(k;\mu)\cdot\underbrace{
    \int_{-\infty }^{+\infty}\mathrm{d}q_1\cdot S(q_1)
    \cdots \int_{-\infty }^{+\infty}\mathrm{d}q_k\cdot S(q_k)
    }_{k}\\
    &=\pi(k;\mu)\cdot\underbrace{S(q)\otimes\cdots\otimes S(q)}_{k}\enspace.
    \label{eq:kpe-charge}
\end{align}

其中 $S_q$ 为单光电子电荷谱概率密度函数。傅里叶变换 $\mathcal{F}$ 能够将卷积变为另一个域上的直积:
\begin{equation}
    \mathcal{F}\left[f(t)\otimes g(t)\right]=\mathcal{F}[f(t)]\cdot\mathcal{F}[g(t)]
    \label{eq:fourier}
\end{equation}

因此对~\eqref{eq:kpe-charge} 做傅里叶变换并对 $k\in\mathbb{N}^{+}$ 求和即得正整数个光电子时的电荷谱密度为:
\begin{equation}
    \tilde{S}_{\ge1\text{PE}}(p)=\sum_{k=1}^{\infty}
    \pi(k;\mu)\cdot\tilde{S}^k(p).
    \label{eq:pe-charge}
\end{equation}

考虑~\eqref{eq:pe-charge} 补全 0 个光电子时的无穷级数求和:
\begin{align}
    \tilde{S}_{\ge1\text{PE}}(p)
    &=\left[\sum_{k=0}^{\infty}\pi(k;\mu)\cdot\tilde{S}^k(p)\right]-\pi(k;\mu)\\
    &=\left[\sum_{k=0}^{\infty}\frac{e^{-\mu}}{k!}\cdot\left(\mu\tilde{S}(p)\right)^k\right]-e^{-\mu}\\
    &=e^{\mu(\tilde{S}(p)-1)}-e^{-\mu}\\
    &=e^{-\mu}(e^{\mu\tilde{S}(p)}-1).
    \label{eq:postive-charge}
\end{align}

通过傅里叶变换,能够将任意多个光电子的卷积分布转化为另一个定义域上的直积,
并借助泊松分布的性质实现解析求和,既不需要截断无穷级数求和,也不需要计算卷积,
是完全无损失的理论方法,因此得到的参数应是无偏的。

考虑 0 电荷处的概率密度则需要综合考虑 0 个光电子与单光电子中 Tweedie 成分的贡献:
\begin{align}
    S_{\text{total}}(q=0)
    &=\left[\sum_{k=1}^{\infty}\pi(k;\mu)\cdot S^k(q=0)\right]+\pi(0;\mu)\\
    &=\sum_{k=0}^{\infty}\pi(k;\mu)\cdot S^k(q=0)\\
    &=\sum_{k=0}^{\infty}\frac{e^{-\mu}}{k!}\cdot\left[\mu S(q=0)\right]^k\\
    &=\sum_{k=0}^{\infty}\frac{e^{-\mu}}{k!}\cdot\left[\mu(1-p_0)e^{-\lambda}\right]^k\\
    &=e^{\mu\left[(1-p_0)e^{-\lambda}-1\right]}.
    \label{eq:zero-charge}
\end{align}

\subsection{复合 Simpson 数值积分}
借助~\ref{sec:dft},一组参数已经能够利用 FFT 与 IFFT 映射到一系列采样点上的概率密度函数值。
为了得到原电荷谱上特定区间内的预测频数,需要使用数值积分方法。
常用的数值积分方法包括:矩形法,Lagrange 插值法与 Newton-Cotes 插值法等。

复合 Simpson 积分法是 Newton-Cotes 插值法的特殊形式,特别适用于间隔给定的特定区间上的数值积分。
其具体计算方式如下:
\begin{align}
    \int_{a}^{b}f(x) dx
    &=\sum_{i=1}^{n/2}\frac{h}{3}\left[f(x_{2i-2})+4f(x_{2i-1})+f(x_{2i})\right]-\frac{f^{(4)}(\xi_i)}{90}h^5 \\
    &=\frac{h}{3}\left[f(x_0)+4f(x_1)+2f(x_2)+4f(x_3)+2f(x_4)+\cdots+2f(x_{n-2})+4f(x_{n-1})+f(x_n)\right]
    -\frac{h^2}{90}\sum_{i=1}^{n/2}f^{(4)}(\xi_i) \\
    &=\frac{h}{3}\left[f(x_{0})+4\sum_{i=1}^{n/2}f(x_{2i-1})+2\sum_{i=1}^{n/2-1}f(x_{2i})+f(x_{n})\right]
    -\frac{h^2}{90}\sum_{i=1}^{n/2}f^{(4)}(\xi_i).
    \label{eq:simpson}
\end{align}

其中 $x_i$ 为等间距 $h$ 的序列满足 $a=x_0<x_1<\cdots<x_n=b$。
考虑连续函数 $f^{(4)}(\xi)$ 的介值定理,应有
$\frac{2}{n}\sum_{i=1}^{n/2}f^{(4)}(\xi_i)=f^{(4)}(\xi),\ \xi\in\left[\min(\xi_i), \max(\xi_i)\right]$,
并给出~\eqref{eq:simpson} 的误差项为 $-\frac{nh^5}{180}f^{(4)}(\xi)=-\frac{(b-a)h^4}{180}f^{(4)}(\xi)$,
与原函数的四阶导数成正比,因此当且仅当对于三次及以下多项式,其误差严格为 0,或称为代数精度为 3。

复合 Simpson 积分法只需要使用特定点的函数值,在细分区间较多、每段区间较平滑的函数上,它的表现良好,
因此在本实验中采样点函数值到原电荷谱区间频数的预测均使用复合 Simpson 积分法计算。
为了尽可能减小误差,需要采样点间隔 $h$ 尽可能小。
Freedman–Diaconis 准则给出的区间宽 $h\propto n^{-1/3}$,
对于统计量越大的电荷刻度数据,原电荷谱的分区间数越大、间距越小,复合辛普森积分法效果越好。

\subsection{傅里叶变换的准确度与性能}

计算机并不能够对连续的 $S(q)$ 函数进行傅里叶变换,只能够使用有限离散傅里叶变换(Discrete Fourier Transformation,简称 DFT)对其进行近似,
并使用逆有限离散傅里叶变换(Inverse Discrete Fourier Transformation,简称 IDFT)。
应当从准确度与性能方面进行考量。

DFT 与 IDFT 结合的方法能够在离散距离趋近于无穷小时的极限为连续傅里叶变换与其逆变换,因此当采样间隔足够小时,即可实现理想的近似。
合适的间隔应当由~\ref{theo:shannon} 决定:
\begin{theorem}[Nyquist–Shannon 采样定理]\label{theo:shannon}
    对于具有带宽限制(即傅里叶变换后频率在有限区域以外为零)例如频率上限为 $f_s$ 的连续信号,使用周期冲激序列进行采样,
    若要完全还原初始连续信号的信息而不发生混叠,则应以不低于 $2f_s$ 的频率进行采样。
\end{theorem}

在实际中,由于连续函数的傅里叶变换常常不具有带宽限制,可以设定频率阈值下限,只需要满足采样频率至少高于该阈值对应频率的二倍,
原始连续信号的主要信息就能够被还原。在本研究中,将依据 Freedman–Diaconis 准则划分的电荷谱区间继续细分至 16 份后,
再增加区间已经没有可以显著观测的变化,可以认为已经理想地还原了原始连续电荷谱的信息。

快速傅里叶变换(Fast Fourier Transformation,简称 FFT,逆算法同理简称 IFFT)是一种 DFT 的高效算法,利用了 DFT 中系数的对称性与周期性,
能够将朴素 DFT 算法的 $O(n^2)$ 时间复杂度降低到 $O(n\log{n})$,对于本研究中每个 MCP-PMT 数万至数十万的统计量表现出显著的优势。

综上所述,傅里叶变换刻度方法的本质优势为避免卷积与无穷级数求和的时间复杂度与精度损失,
而使用较多的采样点,同时满足数值积分、FFT 与 IFFT 准确度的要求,
只需要付出较为廉价的内存代价,是理想的升级。

\section{联合刻度}

\subsection{评价指标:BIC}

\subsection{入射光强与角度对单光电子谱的影响}
