% !TeX root = ../thuthesis-example.tex

\chapter{结合波形分析的迭代联合刻度方法}

\section{波形分析方法}
探测器中发生的事件光学或带电粒子信号完全依靠 PMT 读出,当短时间(例如单光电子电压波形半高宽时间窗)内只有一个光电子到达 PMT,
则在相同的工作条件下,不同 PMT 具有相同的增益数量级,都能输出形状相似的负极性脉冲。然而,当至少两个光电子到达时,
各光电子响应的电压波形将堆积,使得光电子的个数与各自的能量、到达时间分辨难度提高。
面对该问题,有两种常用的技术手段:
\begin{enumerate}
    \item 以光电子波形到达前的一段波形计算本底信号的输出电压水平(基线),并将光电子波形积分得到电荷信息,
    使用经验换算关系得到光电子数并反推能量关系,时间信息则取波峰的特定上升沿(通常为 10\%)时刻为到达时间;
    \item 使用经验的单光电子波形,与去除基线的波形做反卷积,得到各光电子的到达时间。
\end{enumerate}

对于方法 1,由于前一个时间窗口的靠后位置可能存在信号脉冲,或脉冲的后延平稳电压水平与基线不同,电荷积分的区域不容易
形成一致共识。同时,每个光电子倍增后的电荷也具有概率分布,而方法 1 只能通过经验关系换算得到光电子数,无法得到其期望与方差,
因此对提高能量分辨率没有帮助。而对于时间分辨率,由于不同光电子波形的堆叠,方法 1 没有办法准确地区分出各个光电子的到达时间,
从而在事例重建的位置分辨率上也存在客观缺陷。

对于方法 2,对于不同的 PMT 与不同的工作电压,由于电场、渡越速度、噪声本底的水平不同,单光电子波形的幅值与展宽均有差异。
如果全部使用经验单光电子波形进行反卷积,得到的光电子数量与时间将分别由幅值与展宽的有偏估计引入偏差,从而减低事例重建的
能量分辨率与位置分辨率。

为了实现事例重建的能量分辨率与位置分辨率提升,需要一个贝叶斯方法,使得能够在不同的单光电子数量、到达时间、电荷样本空间
(这些空间是维度可变的)与单光电子响应波形找到最优解并给出误差分析。
\subsection{快速随机匹配追踪算法}
快速随机匹配追踪算法(Fast Stochastic Matching Pursuit,以下简称 FSMP)\cite{wangFastStochasticMatching2024a}
是一种可逆跳跃的马尔可夫链蒙特卡罗方法(Reversible Jump Markov Chain Monte Carlo),能够在不同维度的样本空间内采样,
利用 Metropolis-Hastings 等采样方法,实现采样结果的跳跃(马尔可夫链状态的转移),最终收敛至马尔可夫链的稳态分布。

针对打拿级 PMT,该方法认为单光电子电荷服从高斯分布,光电子数服从泊松分布,因此波形中电荷的概率分布模型为复合泊松-高斯分布。
针对 MCP-PMT,该方法讲单光电子电荷模型分解为若干个高斯分布线性之和,用以表示光电子在涂层表面不同行为的分类,
每一种高斯分布具有各自独立的期望与方差。
基于~\ref{sec:spe-charge} 中所述 Gamma-Tweedie 混合电荷模型,仍需要找到使用若干个高斯分布线性之和尽可能近似该分布的方法。

该方法在使用单高斯电荷响应的前提下,于模拟数据集与 8 寸 MCP-PMT 激光测试的实际数据集上均得到了应用与验证,已经获得了能量与时间分辨率的提升。
该工作给出结论,在兆电子伏特的液体闪烁体中微子探测中,该方法在理想情况下,可以提高可见能量的分辨率12\%。

\subsection{波形分析迭代}
在本研究中,FSMP 使用 JWAPtool(Jinping Waveform Advanced Preanalysis tool,用以分析波形的基线、白噪声方差、电荷、峰个数、峰时间等信息)
对每个通道筛选的单光电子波形作为该通道 PMT 波形分析的先验输入,并对电荷做唯象的双 Gamma 拟合得到主峰的均值与方差
作为该 PMT 的高斯电荷模型参数,FSMP 即能够达到研究\cite{wangFastStochasticMatching2024a}中对 MCP-PMT 应用的效果,
获得第一轮分析的光电子到达时刻以及电荷。



\section{联合刻度}

\subsection{评价指标:BIC}

\subsection{入射光强与角度对单光电子谱的影响}

\section{波形分析迭代}

\subsection{}
