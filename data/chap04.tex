% !TeX root = ../thuthesis-example.tex

\chapter{结果}

\section{工作总结}
本论文将 8 寸 MCP-PMT 的电荷模型应用在 JUNO 等中微子探测实验广泛应用的 20 寸 MCP-PMT 上,
证实该电荷模型具有较好的适用性。

为了提高电荷长尾拟合的准确率以及赋予拟合方法在高光强工作条件下的适用性,论文推导得到了任意光电子数目的电荷谱响应,
基于傅里叶变换变换得到了能够适应不同光强的拟合方法,除了提高激光数据增益刻度的准确性,
也为该 MCP-PMT 在高光强场景下的应用提供了理论无偏的拟合工具。

论文基于 FSMP 波形分析方法得到电荷谱,并找到了由电荷模型近似极大似然高斯混合模型解的方法,
在 OSIRIS 探测器上实现了各 MCP-PMT 的增益刻度与高斯混合电荷模型近似,
使得波形分析方法迭代成为现实。

\section{问题与改进方向}
论文还需要进行多轮迭代,直到达到参数收敛的标准。

论文基于 FSMP 波形分析方法得到了光电子到达时间的归一化曲线,即光变曲线,
能够以此进行单光电子波形的重新筛选以得到更准确的波形分析结果。
以后的工作将着力于实现该想法。
